\newpage
\section{SuperSafeApp}

Para la segunda parte del Trabajo Práctico desarrollamos la aplicación \emph{SuperSafeApp} que funciona como un Remote Access Tool (RAT), con el objetivo de lograr lo pedido en el enunciado:

\begin{itemize}
\item Obtener contactos (y toda su información)
\item Obtener el registro de llamadas
\item Obtener el registro de llamadas
\item Obtener los mensajes
\item Ubicación del GPS/RED
\item Tomar una foto con la cámara
\item Capturar el audio con el micrófono
\item Enviar mensajes de texto
\item Abrir una URL en el browser
\item Hacer que el teléfono vibre.
\item Descargar un binario dinámicamente y ejecutarlo.
\end{itemize}

Por otra parte, implementamos una nueva aplicación firmada con el mismo par de claves que la utilizada por la aplicación que implementa el RAT, para que puedan comunicarse, y sumar sus capacidades, realizando tareas maliciosas.

Además, SuperSafeApp incluye funcionalidad para actuar como ransomware.

Parar realizar el RAT, el enunciado sugiere ver como ejemplo el código del androrat. Así que buscamos el código en github, pero al verlo rapidamente observamos que tenía cierta complejidad tratar de entender que se estaba haciendo, y decidimos implementar las funcionalidades nosotros mismos, investigando el uso de la funciones provistas por Android parra realizar ciertas acciones como enviar un mensaje desde una aplicación y usando los conocimientos adquiridos durante la materia, lograr la implementación del RAT.

\subsection{Remote Access Tool (RAT)}

A continuación describimos la estructura general de la implementación y funcionamiento.

La idea de un Remote Access Tool es que mediante algún \emph{evento externo} (remoto), la aplicacion pueda escuchar este evento y saber responder al evento, realizando la acción que se requirió remotamente. 

En este caso, decidimos hacer que al enviarle un mensaje de texto a un dispositivo que contenga esta aplicación, con nombre del comando a ejecutar en el cuerpo del mensaje, la aplicación se encarga de leer ese mensaje cuando llega, interpretar cual es el comando a ejecutar, y ejecutarlo efectivamente.

Un problema que tuvimos en el uso de los mensajes de texto, además de que requiere permisos especiales, fue la prioridad que le asignabamos al
lector de comandos por mensajes. Esto se puede controlar desde la propia actividad, sin necesitar permisos, con lo cual fue un problema de implementación unicamente. Sin embargo, nos pareció el mejor vector de ataque para este tipo de aplicaciones ya que cualquier celular con
android puede enviar y recibir mensajes de texto, y además podíamos embeber los comandos en texto ruidoso (por ejemplo, texto que simulase ser
una oferta para un automóvil, como suelen recibir seguido muchos usuarios de telefonos celulares) de manera que pasaran más desapercibidos.

Otros mecanismos, nos pareció, también sufren del problema de identificar al telefono mismo. SMS no sufre de este problema puesto que con
el número alcanza para enviar un mensaje a un telefono en particular. Otras posibilidades para experimentar serían por ejemplo utilizar WebSockets (ya que la mayoría de los celulares android tiene navegadores modernos), GETs por HTTP, etc. Sin embargo, no nos parecieron en su momento opciones adecuadas o que tuvieran una ventaja muy superior a mensajes de texto. Donde si ganan es en el area permisos, ya que nos parece que es mucho más probable que un usuario permita a una aplicación acceder a internet que a sus mensajes de texto, pero hacen ruido facilmente detectable (porque hay muchos pedidos a un sitio web en particular, por ejemplo si usaramos \textit{polling} por HTTP).

Para poder la interpretación y ejecución de comandos, se definió una interfaz {\tt Commandable} que declara cada uno de los métodos que implementa cada una de las funcionalidades del RAT, llamemos a cada una de estas funcionalidades, \emph{comando}. En este caso, el {\tt MainActivity} implementa esta interfaz, es decir, que en el MainActivity están todos los métodos que implementan las funcionalidades del RAT.

También hay una interfaz {\tt Command}, que tiene un método {\tt execute}, y hay una clase para cada comando, que implementa esta interfaz. 
La idea es que execute se encargue precisamente de ejecutar el comando deseado llamando al método correspondiente dentro del MainActivity.

Por otra parte hay una clase llamada {\tt Command Parser}. Esta clase es la encargada de hacer que se ejecute el comando correcto a partir de una determinada acción, más adelante hablamos de esta acción. Para poder lograr esto, tiene definido el conjunto de comandos aceptados. Estos son:

\begin{itemize}
\item {\tt VIBRATE}
\item {\tt CONTACTS}
\item {\tt PHOTO}
\item {\tt RANSOM}
\item {\tt SENDSMS}
\item {\tt CALLLOG}
\item {\tt LOCATION}
\end{itemize}

Y además implementa el método {\tt dispatch} que dado un string que representa a un comando, según este string, crea una instancia de la clase del comando deseado. En caso de que el string recibido no se corresponda con ningún comando conocido se crea un {\tt UnknownCommand}.

Hasta acá tenemos el funcionamiento de la aplicación una vez recibido y parseado un mensaje de texto que contiene algún comando.
Veamos ahora como hace la aplicación para saber de la llegada de un mensaje de texto y parsearlo.

Para la recepción de comandos utilizamos como punto de llegada un sitio web especial: \url{http://www.tcbpg.com.ar/endpoint.php}. Este
sitio concentra la información que obtenemos del usuario. Para poder desambiguar un usuario del otro los distinguimos mediante el 
\texttt{IMEI} (\textit{International Mobile Equipment Identity}). 

Como deseamos aceptar comandos solamente de nuestra aplicación, utilizamos para ello un protocolo de identificación entre la aplicación
y el servidor. Por simplicidad, decidimos evitar un protocolo posiblemente complejo de \textit{challenge-response} y utilizamos una autenticación
por clave compartida. La clave esta hardcodeada en ambos extremos de la aplicación. El lado del servidor no es un problema, aunque el lado
del cliente podría serlo ya que el código de android esta en posesión del mismo. Por simplicidad de la premisa decidimos utilizar este sistema
(puesto que de todos modos, fortalecer esta parte del esquema no tiene sentido considerando que la aplicación en si misma necesita muchos
permisos para funcionar). 

El mecanismo de autenticación utiliza un HMAC de SHA-1 basado en el payload considerado como combinación de tipo de comando (\texttt{messageType}),
cuerpo del comando (\texttt{messageBody}), imei y un \textit{timestamp}. De esta manera podemos asegurar que un comando vino de nuestra
aplicación. El propósito del timestamp es evitar un \textit{replay-attack}, si bien por motivos de tiempo el servidor en este momento no lo
utiliza (consideramos que el cambio es trivial, pues consiste en utilizar una \textit{query} de SQL para detectar si ya se uso un comando con
ese timestamp).

Para almacenar los datos utilizamos una base de datos MySQL. Los datos son previamente escapeados para evitar
injecciones de SQL, para lo cual utilizamos el mecanismo estándar para hacer \textit{queries} de manera
segura en PHP 5.3: PDO (\textit{PHP Data Objects}). Utilizar \textit{prepared queries} de PDO asegura que
todo dato insertado a la base de datos es higienizado de manera de disminuir el riesgo de un ataque a
la base de datos.

Segun el tipo de comando, el sistema toma una acción distinta. A continuación detallamos las acciones programas y como se programaron las mismas.

\subsubsection{Obtener contactos}

La obtención de contactos es sencilla, pero require un permiso especial de acceso a los
datos de contactos del teléfono (\textit{android.permission.READ\_CONTACTS}). Lo que se hace
entonces es usar un \texttt{ContactsContract} que utiliza un \texttt{ContentResolver} para
realizar los pedidos y crear una lista de contactos. Cada contacto esta representado en una
clase en Java que llamamos \texttt{Contact}. Lo que enviamos entonces es la serialización
de la lista de contactos al servidor.

\subsubsection{Obtener el registro de llamadas}

El registro de llamadas se obtiene con un mecanismo similar al de obtener contactos, y también
enviamos el registro de llamadas como una cadena de caracteres. Extraemos los siguientes datos
para cada entrada del registro de llamadas:

\begin{itemize}
	\item El tipo de llamada: Perdida, llamada desde el telefono o Recibida.
	\item El numero de telefono (distinto al actual) utilizado.
	\item Fecha de la llamada.
	\item Duración de la llamada en segundos.
\end{itemize}

Para enviar los datos, los serializamos en un string de manera que nos sirva como
cuerpo del mensaje que llega al servidor. El formato es fácil de parsear por lo cual
sería sencillo extraer datos importantes de lo que esta guardado en la base de datos
(en este caso, MySQL del lado del servidor).

\subsubsection{Obtener los mensajes}

Este string, al igual que el String resultado de obtener contactos, es enviado al 
servidor y almacenado como esta en la base de datos. Esto permite procesarlos 
utilizando otros scripts.

\subsubsection{Ubicación del GPS/RED}
%Cuando le llega un mensaje al teléfono la app debe escuchar eso (lo hace SMSReceiver). Obtener la ubicación actual del usuario y enviársela al servidor a través de un post.
%
%SMSReceiver funciona como un publish subscribe. 
%Crear una clase para obtener la ubicación, tipo LoactionManager, que se suscriba al SMSReceiver, y que implemente los métodos necesarios para ser notificado de la llegada de un mensaje.
%Entonces, cuando llegue un mensaje, la clase locationManager se va enterar y lo que debe hacer es obtener la ubicación, del usuario guardarla, dejar de actualizar la ubicación, y enviar los datos conseguidos al server a través de un post.
%
%Mi locationmanager debe implementar Command. Esto implica que debe implementar el método execute. A la vez debe tener como variable privada un commandable.
%commendable tiene definidos los métodos a usar. Entonces debería agregar en commandable algo como getUserLocation. Y execute llamará a este método, que estará definido en MainActivity.
%
%En el post enviar en un json:
%-número de imei
%-location

\subsubsection{Tomar una foto con la cámara}

Actualmente esta hecho el comando de tomar fotos con la cámara. Dado que nos pareció que
no aportaba demasiado en si, decidimos simplemente tomar la foto y almacenarla. Utilizando
la otra aplicación creada (\textit{La llama que no llama}) puede obtenerse esta foto en vez
de enviarla al servidor cada vez. Enviar una foto es un proceso algo arduo (como vimos en las
dificultades de La Llama que No Llama) y preferimos concentrarnos en otros aspectos de esta 
aplicación que correspondían a funcionalidades que deseabamos probar.

Para acceder a esta funcionalidad es necesario declarar el permiso
(\texttt{android.permission.CAMERA}) en el Manifest de Android. Esto nos permite
tomar fotos desde el celular.

Si nos interesaba lograr tomarle una foto al usuario sin que este se percatara
de que el telefono lo estaba haciendo. Para ello, pedimos y utilizamos la Camara
sin llamar al Intent de sacar foto (la manera estándar en la que se utiliza la
camara de Android). También utilizamos una superficie no visible para proyectar
el resultado de la cámara (requerimiento para el uso de la cámara mediante la
clase \texttt{android.Camera}). Utilizando un \texttt{CameraCallback} podemos 
tomar los bytes tomados de la imágen y guardarlos en la tarjeta SIM, de manera
que pueda utilizarse otra aplicación para robarla.

Una posible mejora sería enviar la imagen, posiblemente como Base 64 (al ser
esto un string permitiría enviarla utilizando la arquitectura ya pensada para
el resto de los comandos). Nos interesaba en particular hacer este proceso
invisible al usuario y dejamos esto que no nos parecía particularmente interesantes
(pues lo habíamos hechos para otros comandos).

\subsubsection{Hacer que el teléfono vibre}

Este comando fue el primero que implementamos ya que sirvió como prueba base de la 
funcionalidad del recepción de comandos por SMS. Al enviarse el comando, el telefono
vibra por 

\subsection{Ransomware}

Para implementar el \textit{ransomwarer}, lo que buscabamos era escribir un comando al que pudieramos pasarle un parametro, el archivo
a tomar como rehén. Esto es sencillo ya que la infraestructura que implementamos para los comandos anteriores nos permite soportar esta
funcionalidad de manera sencilla, sin cambios mayores.

Por otro lado, queríamos experimentar con maneras de separar la aplicación para que los distintos componentes necesitaran menos permisos
(y de esta manera fuese más probable que un usuario no se percate de que la aplicación es maliciosa). Con este propósito, buscamos separar
la parte de la aplicación que se dedica a tomar como rehén el archivo del usuario de la aplicación principal \textit{SuperSafeApp}.

Para lograr esto, creamos otra aplicación, \textit{Ransomwarer}. Por motivos de simplicidad y propósito, de la misma manera que para
la \textit{SuperSafeApp}, no hicimos una fachada. Nuestro objetivo con esto era experimentar con maneras de ataque y obtención de 
información, y de como enviarla desde la aplicación maliciosa a nuestra posesión.

Si nos intereso, sin embargo, no dar evidencia al usuario de que la aplicación que toma como rehen a un archivo estuviese corriendo. Para
eso lo que programamos para \textit{Ransomwarer} es una aplicación que tiene una fachada (no desarrollada pero que nos sirve para la
prueba de concepto) y un servicio por detrás. El servicio corre todo el tiempo y responde a mensajes con un cierto tipo MIME, que a falta
de mejor nombre decidimos ponerle \textit{ransom/note}. Cuando recibe un mensaje de este tipo, extrae el archivo que indica y luego cifra
el archivo original usando una clave aleatoria, y se deshace del original. De esta manera en el telefono del usuario solo queda el archivo
cifrado como rehén. Posteriormente envia la clave utilizada al servidor, junto con demas datos de identificación, y un SHA1 del archivo
en cuestion.

De acuerdo a la documentación de Android, cualquier Intent que utilize este tipo de MIME sera recibida por el \textit{Ransomwarer} por lo cual
no hay necesidad de cifrar con la misma clave ambas aplicaciones: El usuario ya accedió a tener un servicio corriendo que responde al tipo de
MIME especial que nosotros utilizamos por lo tanto ya podemos comunicar ambas aplicaciones que tenemos de manera sencilla.

El algoritmo de encripción es AES 256 utilizando CBC (\textit{Cypher Block Chaining}), usando un vector de inicialización \textit{hardcordeado}
y una clave aleatoria. El vector de inicialización es fijo pero facilmente podría cambiarse este esquema para utilizar uno aleatorio (basta con
enviarselo al servidor junto con el resto del pedido). El algoritmo fue elegido porque es rápido, seguro, y de cifrado simétrico (con lo cual
no es necesario realizar un proceso de desencripción más complicado y largo, lo cual también impacta positivamente en la performance puesto que
el algoritmo es rápido y se tiene que aplicar sobre archivos potencialmente muy grandes).

Para lograr enviar todos estos datos utilizando la arquitectura de identificación entre el servidor y el
telefóno, enviamos el comando en el cuerpo del mensaje (\texttt{messageBody}) en formato JSON. Este formato
es muy sencillo, estandar y fácil de desencriptar del lado del servidor, y nos ahorramos tener que proteger
de modificación (mantener la integridad) los otros campos, pues podemos usar el mismo mecanismo.

Una vez enviados estos datos, el servidor los almacena junto con un dinero a pagar, tomado aleatoriamente entre 100 y 5000 (la unidad de cambio
se deja abierta). Esto se corresponde con el rescate que el usuario debe pagar para que le devolvamos su archivo.

Una vez enviado y cifrado el archivo, la aplicación avisa que el mismo fue tomado como rehén e indica una página, ubicada en la URL
\url{http://www.tcbpg.com.ar/recovery.php}, donde el usuario puede pagar el monto y recuperar su archivo. Para hacer el rescate, el usuario
indica el IMEI de su telefono, sube el archivo cifrado, el nombre original y el monto que va a pagar. Si paga menos, solo se le indica que pago
menos (de manera de estimular que se sobrepague el rescate, de manera de maximizar nuestras ganancias). Una vez que paga, se le devuelve el
archivo original desencriptado.

El propósito del SHA1 que tomamos originalmente es para identificar al archivo. Puesto que el mismo es parte del payload que se envia al
endpoint, y este ya esta protegido por un HMAC, lo podemos enviar asi como esta porque tenemos seguridad de que esta bien. Luego, para evitar
requests espurias (por ejemplo, subir muchas veces un archivo que no existe) además de las protecciones usuales de tamaño de archivo (que son
parte de la configuración del servidor de Apache que corre nuestro recuperador), usamos el SHA1 para asegurar que sea un rescate genuino.

Una vez rescatado el archivo, borramos la entrada en la base de datos, para evitar ataques usando un mismo archivo repetidamente.

Este es el propósito, adicionalmente, de que sea el cliente el que tenga el archivo cifrado. Ya que de esa manera podemos implementar este
sistema de rescate.

Un punto de complicación importante que tuvimos fue lograr que la aplicación en Android (escrita en Java) y el servidor (escrito en PHP) se
entendieran desde un punto de vista criptográfico: lograr que usaran el mismo IV, la misma configuración, etc. No fue tarea fácil. Si bien
ambos lenguajes poseen una biblioteca de funciones criptográficas de primer nivel, no todos los parámetros están debidamente documentados,
incluso habiendo funciones (por ejemplo en PHP) que están sin documentar. También tuvimos problemas con las dependencias a nivel biblioteca,
ya que por ejemplo PHP depende de \textit{libmcrypt} para que se puedan utilizar las funciones criptográficas que necesitabamos nosotros para
el trabajo práctico.

La programación del servidor no es muy compleja. Empleamos su propósito acotado para tratar de hacerla segura sin recurrir a complicados 
procedimientos (por ejemplo HTTPS) pero manteniendo la misma segura. Esto sigue la política de \textit{minimo privilegio}: Buscamos que SOLO
nuestra aplicación en Android sea la que pueda hablar con el servidor.

También, al desencriptar solamente los archivos que suben los usuarios, tenemos poco problema 
de espacio (lo cual tendríamos si tuvieramos que almacenar todos los archivos en la base de
datos).

%Mezclar ransomware con el de las 2 apps que comparten datos. Para esto habría que hacer una app nueva que sea la que por ejemplo, se conecta a internet
%
%Para hacer la mezcla la idea es que las 2 apps estén firmadas con la misma key. De esta forma pueden las 2 apps pueden intercambiar mensajes entre sí, entonces una hace una cosa y le envía esa información a la otra, y la otra lo manda "por internet".

\subsection{Mejoras}

Las mejoras que podrían realizarse a este sistema que se nos ocurren (hay otras
posibles son):

\begin{itemize}
	\item Agregar nuevos comandos.
	\item Permitir sistemas alternativos de envio de mensajes.
	\item Utilizar un protocolo más fuerte, ya que el actual requiere de una
	clave hardcodeada que podría ser fácilmente descubrible mediante un 
	desensamblado.
	\item Enviar las fotografías tomadas al servidor como Base 64.
	\item Almacenar los datos de una manera mas interesantes (actualmente se
	almacena todo en la base de datos como texto, podría romperse en pedazos
	cada mensaje y almacenarse en tablas separadas).
\end{itemize}

