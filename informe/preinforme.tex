%BEGIN COPYPASTE EL INFORME DEL INFO
\documentclass[10pt, a4paper,english,spanish]{article}
\usepackage{subfig}

\parindent=20pt
\parskip=8pt
\usepackage[width=15.5cm, left=3cm, top=2.5cm, height= 24.5cm]{geometry}

% \usepackage{ccfonts,eulervm} 
% \usepackage[T1]{fontenc}

\usepackage{longtable}
\usepackage{ccfonts,eulervm} 
\usepackage[T1]{fontenc}
% \usepackage{ccfonts,eulervm} 
% \usepackage[T1]{fontenc}
\usepackage{longtable}
\usepackage{epigraph}
\usepackage{amsmath}
\usepackage{amsfonts}
\usepackage{amssymb}
\usepackage[activeacute, spanish,english]{babel}
\usepackage{cancel}
\usepackage[utf8]{inputenc}
\usepackage{algorithm}
%\usepackage{algpseudocode}
\usepackage{afterpage}
\usepackage{caratula}
\usepackage{url}
\usepackage{fancyhdr}
\usepackage{listings}
\usepackage{ulem}
\usepackage{dashrule}
\usepackage{pdflscape}
\usepackage{pgf}
\usepackage{tikz}
\usetikzlibrary{arrows,automata}


\floatname{algorithm}{Algoritmo}

\newtheorem{theorem}{Teorema}[section]
\newtheorem{lemma}[theorem]{Lema}
\newtheorem{proposition}[theorem]{Proposici\'on}
\newtheorem{corollary}[theorem]{Corolario}

\newcommand{\Var}{\textbf{var }}
\newcommand{\True}{\textbf{true }}
\newcommand{\False}{\textbf{false }}
\newcommand{\Break}{\textbf{break }}
\newcommand{\Continue}{\textbf{continue }}
\newcommand{\Param}{\textbf{param }}
\newcommand{\ig}[3]{
	\begin{landscape}
		\begin{figure}[c]
			\label{diag_diseno}
			\includegraphics[scale=#2]{images/#1.pdf}
			\caption{#3}
		\end{figure} 
	\end{landscape}
	\newpage
}




% \parindent 0em
%\algrenewcommand{\algorithmiccomment}[1]{//\textit{#1} }

\renewcommand{\emph}[1]{\textit{#1}}
\pagestyle{fancy}
\thispagestyle{fancy}
\addtolength{\headheight}{1pt}
\lhead{Seguridad - TP1}
\rhead{Grupo 8}
\cfoot{\thepage}
\renewcommand{\footrulewidth}{0.4pt}
\newcommand{\hblacksquare}{\hfill \blacksquare}
%FIN COPYPASTE EL INFORME DEL INFO
\begin{document}

\materia{Seguridad de la Información}
\submateria{Segundo Cuatrim\'estre de 2013}
\titulo{Trabajo Pr\'actico: Malito}
\subtitulo{La Llama que \textbf{no} llama}
\grupo{Grupo ``Queremos sanguiches de miga''}
\integrante{Juli\'an Sackmann}{540/09}{jsackmann@gmail.com}
\integrante{Juan Pablo Darago}{272/10}{jpdarago@gmail.com}
\integrante{Vanesa Stricker}{159/09}{vanesastricker@gmail.com}
% \integrante{Mat\'ias Barbeito}{179/08}{matiasbarbeito@gmail.com}

\maketitle
\pagebreak

\tableofcontents
\pagebreak

\section{Introducción} % (fold)
\label{sec:introducci_n}

La primer parte del TP requiere implementar un \emph{malware} para \texttt{Android} que, al instalarse, envíe las fotos del teléfono a las que tenga acceso a un servidor controlado por nosotros. Para esto, es necesario tomar una serie de decisiones, que se detallan a continuación:

\subsection{Façade} % (fold)
\label{sub:fa_ade}
Si bien no tenemos datos certeros de esto, creemos que es extremadamente poco probable que el usario promedio de \texttt{Android} instale voluntariamente una aplicación que se publicite a si mismo como ``TeRoboLasFotosApp''. Es por esto que consideramos necesario \textbf{disfrazar} el app, de tal forma que la probabilidad de una instalación se incremente ligeramente. 

Dado que no queremos q

% subsection fa_ade (end)
% section introducci_n (end)


\addcontentsline{toc}{section}{Referencias}
\begin{thebibliography}{8}
\raggedright

\bibitem{AndroidMalware}
	El librito ese.
	\newblock {Re loco}.
	\newblock De colores.

\end{thebibliography}

\end{document}
